%&lualatex

\documentclass[
  12pt,
  a4paper,
  oneside,
  %parskip=half,
  parskip=relative,
  numbers=enddot,
  headings=optiontoheadandtoc,
  toc=listof,
]{scrreprt}
\usepackage{scrlayer-scrpage}
\clearpairofpagestyles
\automark[section]{chapter}
\ihead{\headmark}
\cfoot[\pagemark]{\pagemark}
\usepackage{fontspec}
\usepackage{polyglossia}
\setmainlanguage{polish}
%\setmainfont{Liberation Serif}
\setmainfont{arial}
%\usepackage{indentfirst}
\usepackage{setspace}
\usepackage{enumitem}
\usepackage{bm}
\usepackage{amssymb}
\let\gtrapprox\undefined
\usepackage{mathtools}
\usepackage{siunitx}
\usepackage{physics-patch}
\usepackage{graphicx}
\usepackage{float}
\usepackage{listings}
\usepackage{xcolor}
\usepackage{svg}
\usepackage{booktabs}
\usepackage{algorithm}
\usepackage{algorithmic}
\usepackage[
  backend=biber,
  style=numeric,
  sorting=none,
  maxbibnames=6,
  minbibnames=1
]{biblatex}
\usepackage[
  hidelinks
]{hyperref}

%\usepackage[textsize=small]{luatodonotes}
\usepackage[disable]{luatodonotes}

\AtBeginDocument{\RenewCommandCopy\qty\SI}

\sisetup{
  range-phrase = {--},
}

\graphicspath{{Pictures/}}

\addbibresource{bibliografia.bib}
\newtheorem{definition}{Definicja}[section]

\SetLanguageKeys{polish}{indentfirst=true}
\onehalfspacing
\setlength{\parindent}{1.25cm}
\setlength{\parskip}{0pt}

\begin{document}

\begin{titlepage}
\begin{center}
  \hbox{}
  {\Large %\bfseries  
    Uniwersytet Mikołaja Kopernika   \\
    Wydział Matematyki i~Informatyki \\
  }
  \vspace{2.0 cm}
  {\large %\bfseries
    Daniel Nadolny \\
  }
  {\normalsize \mdseries
    nr albumu: 312887 \\
  }
  \vspace{1.2 cm}
  {\normalsize \mdseries
    Praca inżynierska \\
    na kierunku informatyka \\
  }
  \vspace{1.8 cm}
  {\Huge %\bfseries 
    Ray Tracing w czasie rzeczywistym
  }\\
  \vspace{1.4 cm}
  \noindent
  \hspace{0.5\linewidth}
  \begin{minipage}{0.5\linewidth}
    Opiekun pracy dyplomowej \\
    doktor Jakub Narębski   \\
    Wydział Matematyki i~Informatyki
  \end{minipage}\\
  \vspace{3.0 cm}
  {\large 
    Toruń 2026
  }\\
\end{center}
\thispagestyle{empty}
\end{titlepage}

\newpage
\thispagestyle{empty}
\mbox{}
\newpage

\pagenumbering{arabic}
\setcounter{page}{1} % Rozpoczęcie numerowania od tego miejsca
\tableofcontents

\newcommand{\point}[1]{\bm{#1}}
\newcommand{\vecbm}[1]{\bm{#1}}

\chapter*{Wstęp}
\addcontentsline{toc}{chapter}{Wstęp}
\markboth{Wstęp}{Wstęp}
\label{chap:wstep}


W 2018 roku NVIDIA przedstawiła światu nową generację kart graficznych nazwanych RTX. Od tego roku każda kolejna seria począwszy od serii 20 do teraz (seria 50) jest wyposażona w tzw. RT Cores. Rdzenie RT są specjalnie stworzone do przyspieszania obliczeń związanych ze śledzeniem promieni (dalej będę używał nazwy ray tracing), w szczególności przy testowaniu przecięcia promienia z trójkątem i przechodzenia przez strukturę danych zwaną BVH (ang. bounding   volume hierarchy). Odpowiednikiem RT Cores w kartach graficznych od AMD są "Ray accelerators". Ray tracing jest   
bardzo wymagającym algorytmem pod względem obliczeniowym. Dodanie powyższych rozwiązań sprzętowych do GPU pozwoliły programistom implementowanie ray tracingu w czasie rzeczywistym np. w grach, gdzie  obecnie w jednej scenie może pojawić się kilka milionów trójkątów (dotychczas ray tracing wykorzystywany był głownie w filmach).  


W ramach pracy inżynierskiej stworzony został silnik graficzny przedstawiający ray tracing w czasie rzeczywistym, napisany jest w języku C++, wykorzystując bibliotekę DirectX 11 i win32. Interfejs użytkownika stworzony został za pomocą biblioteki ImGui. 


Pierwszy rozdział tej pracy będzie poświęcony przedstawieniu podstaw ray tracingu, od opisania idei algorytmu do wyprowadzenia dwóch podstawowych procedur badania przecięć promienia z obiektami w scenie (promień-sfera i promień-trójkąt). W tym rozdziale poruszone będzie również zagadnienie optymalizacji silnika używając struktury danych BVH. W następnym rozdziale zostanie opisane zagadnienie materiałów. Materiały są jednym z najważniejszych tematów w ray tracingu, jak i ogólnie w grafice komputerowej, jeśli chodzi o aspekty wizualne.
W trzecim rozdziale opisany będzie stworzony silnik i implementacje przedstawionych wcześniej algorytmów. W końcowej części pracy przedstawione zostaną wyniki testów wydajnościowych programu. Testy były przeprowadzone przed optymalizacją silnika i po optymalizacji, aby wykazać różnice wydajności. 
\chapter{Podstawy Ray Tracingu}
\label{chap:podstawy}

\section{Definicje}
\chapter{Optymalizacja - Bounding Volume Hierarchy}
\label{chap:optymalizacja}

\section{Optymalizacja przecinania promienia z trójkątem}

Wydajność renderowania scen za pomocą ray tracingu zależy od kilku parametrów: rozdzielczości obrazu, liczby odbić promienia, liczby promieni na piksel i liczby trójkątów w scenie. Wzrost tych wartości przekłada się na wyższą szczegółowość sceny, jednak zdecydowanie zwiększa koszt jej wyrenderowania. Głównym polem optymalizacji jest minimalizacja liczby testów przecięcia promień-trójkąt. Do takiej optymalizacji można posłużyć się tzw. drzewami BVH -- Bounding Volume Hierarchy.

Idea algorytmu jest prosta - pomijać te trójkąty których promień na pewno nie przetnie. Bez optymalizacji, aby sprawdzić czy promień przecina się z jakimś trójkątem, trzeba przejść przez całą tablicę trójkątów modelu i każdy przetestować, w przypadku użycia BVH większość pomijamy badając tylko te najbliższe punktowi przecięcia.

\section{Zasada działania BVH}
W BVH dzielimy dany model na prostopadłościany zwane ,,bounding box". Proces podziału rozpoczyna się od korzenia (ang. root) jako prostopadłościan okalający cały obiekt (lub całą scenę), następnie rekurencyjnie dzielimy model na coraz to mniejsze prostopadłościany, aż do zadanej granicy np. 2 trójkątów w jednym boxie. BVH ma strukturę drzewa binarnego (tak jest w stworzonym silniku), gdzie w wierzchołkach mieszczą się kolejne prostopadłościany z podziału, a w liściach znajdują się trójkąty \cite{realTimeRendering}. 

\begin{figure}[tbph]
    \centering
    \includegraphics[width=1\textwidth]{BVH2} 
    \caption{Obrazek przedstawiający BVH. Podpunkt $a$ przedstawia objętości (boxy), a w nich obiekty. Podpunkt $b$ pokazuje drzewo BVH stworzone na podstawie sceny z podpunktu $a$~\cite{pbrbook}.}
    \label{fig:BVH}
\end{figure}

W strukturze BVH używa się różnych typów ,,bounding box", w ray tracingu popularnym wyborem są ,,axis-aligned bounding box" (AABB). AABB tworzone są poprzez wyznaczenie dwóch punktów: prawego górnego punktu i lewego dolnego punktu. Maksymalny punkt AABB wyznaczany jest poprzez wyszukanie największych wartości na każdej osi ze zbioru trójkątów, każdy trójkąt ma 3 punkty. Punkt minimalny wyznaczany jest analogicznie poprzez szukanie najmniejszej wartości.

Pseudokod algorytmu wyznaczania prostopadłościanu ograniczającego (axis aligned bounding box) dla zbioru trójkątów pokazany jest jako algorytm.~\ref{algo:aabbAlgo}. 

\begin{algorithm}
\caption{Wyznaczanie AABB}
\label{algo:aabbAlgo}
\begin{algorithmic}
\STATE Triangles $\gets$ [ $(v_1, v_2, v_3)$, $\ldots$ ]
\STATE $\point{p_{\text{min}}} \gets (\infty, \infty, \infty)$
\STATE $\point{p_{\text{max}}} \gets (-\infty, -\infty, -\infty)$
\FOR{each triangle $t$ in Triangles}
    \STATE $\point{p_{\text{min}}} \gets \bigr( \min(\bm{P_{min}}, \bm{t.v_1})) $
    \STATE $\point{p_{\text{min}}} \gets \bigr( \min(\bm{P_{min}}, \bm{t.v_2})) $
    \STATE $\point{p_{\text{min}}} \gets \bigr( \min(\bm{P_{min}}, \bm{t.v_3})) $
    \STATE $\point{p_{\text{max}}} \gets \bigr( \max(\bm{P_{max}}, \bm{t.v_1})) $ 
    \STATE $\point{p_{\text{max}}} \gets \bigr( \max(\bm{P_{max}}, \bm{t.v_2})) $
    \STATE $\point{p_{\text{max}}} \gets \bigr( \max(\bm{P_{max}}, \bm{t.v_3})) $
\ENDFOR
\end{algorithmic}
\end{algorithm}

\section{Konstrukcja drzewa BVH}

Aby stworzyć strukturę drzewa należy teraz podjąć decyzję w jaki sposób dzielić model (AABB boxy). Dla prostoty algorytmu dzielić można wzdłuż najdłuższej osi \cite{realTimeRendering}. Następnie przypisujemy trójkąty do jednego lub drugiego poddrzewa BVH poprzez sprawdzanie czy jego centroid \footnote{Centroid to punkt przecięcia median trójkąta (mediana to odcinek łączący wierzchołek ze środkiem przeciwległego boku) \cite{centroidWIki}} leży po jednej czy drugiej stronie podziału.

\begin{figure}[tbph]
    \centering
    \includegraphics[width=0.8\textwidth]{Centroid} 
    \caption{Obrazek przedstawiający centroid trójkąta~\cite{centroidWIki}.}
    \label{fig:centroid}
\end{figure}


\section{Przechodzenie przez drzewo BVH}

Najważniejszym punktem w procedurze przechodzenia przez drzewo BVH jest testowanie przecięcia promienia z AABB. W tym przypadku test nie musi zwracać dokładnego punktu przecięcia, ale samą informację, czy do przecięcia doszło lub (jeśli potrzebna) odległość od punktu początkowego promienia do przecięcia. Jedną z metod jest tzw. slab method. 


Slab test polega na sprawdzaniu czy promień przecina wszystkie płaszczyzny wyznaczane przez osie x, y, z.
Dla każdej osi należy obliczyć parametr $t$, który wyznacza możliwy punkt przecięcia~\eqref{eq:ray}. AABB box zdefinowany jest jako dwa punkty $\bm{P_{min}} = (x_{min}, y_{min}, z_{min})$, $\bm{P_{max}} = (x_{max}, y_{max}, z_{max})$, jak na rysunku~\ref{fig:aabbPic}.


\begin{figure}[tbph]
    \centering
    \includegraphics[width=0.8\textwidth]{AABB} 
    \caption{Obrazek przedstawiający AABB~\cite{aabb}.}
    \label{fig:aabbPic}
\end{figure}

Wzory na parametr $t$:
\begin{align}
    t_{1} &= \frac{\bm{P_{min}} - \bm{O}}{\bm{D}} \\
    t_{2} &= \frac{\bm{P_{max}} - \bm{O}}{\bm{D}}
\end{align}

Następnym krokiem jest wybranie wartości największej i najmniejszej:
\begin{align}
    t_{min} &= \min(t_1, t_2) \\
    t_{max} &= \max(t_1, t_2)
\end{align}

Następnie należy obliczyć wartości: 
\begin{align}
    t_{near} = \max(t_{min_{x}}, t_{min_{y}}, t_{min_{z}}) \\
    t_{far} = \min(t_{max_{x}}, t_{max_{y}}, t_{max_{z}})
\end{align}

Jeśli $t_{near} <= t_{far}$ to promień jest na części wspólnej wyznaczonej przez płaszczyzny, czyli przecina AABB box, jeśli $t_{near} >= t_{far}$ lub $t_{far} < 0$ promień nie trafił w box.


Jeśli promień przetnie box AABB, wtedy trzeba sprawdzić czy wierzchołek ze struktury BVH jest liściem. Jeśli tak (liść ma w sobie trójkąty), to uruchamia się test przecięcia promień-trójkąt. W przypadku wierzchołków, które nie są liśćmi, funkcja wchodzi w rekurencje dla lewego potomka i prawego potomka.


Podział nie zawsze jest optymalny, jak pokazuje rysunek \ref{fig:bvhSplit}. W podpunkcie $b$ widać, że AABB nachodzą się co powoduje, że sprawdzając przecięty trójkąt trzeba przejść przez dwóch potomków. Dlatego opracowano również inne metody.

\begin{figure}[tbph]
    \centering
    \includegraphics[width=0.8\textwidth]{BVHPodzial} 
    \caption{Obrazek przedstawiający podział na podstawie punktu środkowego centroidów na osi~\cite{pbrbook}.}
    \label{fig:bvhSplit}
\end{figure}

\section{Surface Area Heuristic}

W tej metodzie przy konstrukcji każdego podziału oblicza się koszt przeprowadzania testów przecięcia promienia z danym obiektem. Wzór na koszt \cite{pbrbook}:
\begin{equation}
    c(A,B) = t_{trav} + p_A\sum_{i=1}^{N_A} t_{isect}(a_i) + p_B\sum_{i=1}^{N_B}t_{isect}(b_i)
\end{equation}

\begin{itemize}
    \item $t_{trav}$ - czas potrzebny na ustalenie przez które dzieci przechodzi promień.
    \item $p_A$ - to prawdopodobieństwo, że przez wierzchołek A przejdzie promień (odpowiednio $p_B$).
    \item $\sum_{i=1}^{N_A} t_{isect}(a_i)$ - Czas potrzebny na przeprowadzenie testu przecięcia promienia z obiektem dla każdego obiektu w danym AABB.
\end{itemize}

\begin{equation}
    p_A = \frac{s_a}{s_b}
\end{equation}
\begin{itemize}
    \item $s_a$ - powierzchnia danego AABB ($s_a < s_b$)
\end{itemize}

Wzór opisuje prawdopodobieństwo przecięcia powierzchni $A$ jeśli przecięta została powierzchnia $B$, rysunek~\ref{fig:bvhAB}


\begin{figure}[tbph]
    \centering
    \includegraphics[width=0.8\textwidth]{ABsah.png} 
    \caption{Rysunek przedstawiający powierzchnię A i B}
    \label{fig:bvhAB}
\end{figure}
\chapter{Materiały i modele oświetlenia}
\label{chap:materialy}

Zachowanie światła padającego na dany obiekt będzie zależało od materiału z którego ten obiekt jest stworzony. Na przyład światło padające na metal będzie odbijało się inaczej od światła padającego na plastik. W grafice komputerowej chcemy symulować właściwości różnych materiałów za pomocą tzw. modeli oświetlenia.


Model oświetlenia, to matematyczny opis zachowania światła w scenie. Określa w jaki sposób obiekty powinny być renderowane tj. w jaki sposób światło odbija się od powierzchni obiektów. Modele oświetlenia możemy podzielić na~\cite{realTimeRendering}:
\begin{itemize}
    \item Modele oświetlenia lokalnego (local illumination) - kolor danej powierzchni zależy tylko od materiału, z którego powierzchnia jest zrobiona i źródła światła.
    \item Modele oświetlenia globalnego (global illumination) - kolor danej powierzchni zależy od materiału, z którego powierzchnia jest zrobiona, źródła światła i światła odbitego od innych obiektów.
\end{itemize} 


\section{Model oświetlenia lokalnego}

Jednym z najpopularniejszych modeli oświetlenia lokalnego jest model Phonga (ang. Phong reflection model). Model Phonga~\cite{phong} składa się z trzech komponentów:
\begin{itemize}
    \item Ambient - światło otoczenia. W tym algorytmie wpływ innych obiektów na jasność i kolor danego modelu (global illumination) jest symulowane poprzez dodanie pewnej stałej wartości do jasności i koloru obiektu.
    \item Diffuse - światło rozproszone. Najważniejszy komponent oświetlenia, światło padające na powierzchnię odbija się we wszystkich kierunkach równomiernie \cite{pbrbook}
    \item Specular - światło zwierciadlane, odblask.
\end{itemize}

\begin{figure}[tbph]
    \centering
    \includegraphics[width=1\textwidth]{PhongKomponenty} 
    \caption{Obrazek przedstawia komponenty tworzące model phonga~\cite{phongWiki}}
    \label{fig:phongComp}
\end{figure}

Wzór przedstawiający ostateczne oświetlenie danego punktu:

\begin{figure}[tbph]
    \centering
    \includegraphics[width=1\textwidth]{Blinn_Vectors} 
    \caption{Wektory w modelu phonga~\cite{phongWiki}}
    \label{fig:phongVec}
\end{figure}

\begin{equation}
    I = I_{a}k_{a} + k_{d}I_{i}(\bm{n} \cdot \bm{l}) + k_{s}I_{i}(\bm{r} \cdot \bm{v})^s
\end{equation}
%
Gdzie: 
\begin{itemize}
    \item $I_{a}$ - natężenie światła otoczenia.
    \item $I_{i}$ - natężenie światła padającego.
    \item $k_{a}$ - współczynnik odbicia światła otoczenia. Własność materiału.
    \item $k_{d}$ - współczynnik odbicia rozproszonego. Określa jak bardzo obiekt jest matowy. Własność materiału.
    \item $k_{s}$ - współczynnik odbicia zwierciadlanego. Określa jak bardzo obiekt jest błyszczący. Własność materiału. 
    \item $\bm{n}$ - wektor normalny do powierzchni.
    \item $\bm{l}$ - znormalizowany wektor kierunku do źródła światła od badanego punktu.
    \item $\bm{r}$ - wektor kierunku odbicia światła obliczany według wzoru: \[r = 2.0(\bm{l} \cdot \bm{n}) \cdot \bm{n} - \bm{l} \]
    \item $\bm{v}$ - znormalizowany wektor kierunku do kamery.
    \item $s$ - współczynnik przedstawiający rozmiar odblasku.
\end{itemize}

W przypadku wielu źródeł światła w scenie wzór przyjmuje postać: 

\begin{equation}
    I = I_{a}k_{a} + \sum_{i}^{lights}k_{d}I_{i}(\bm{n} \cdot \bm{l_{i}}) + \sum_{i}^{lights}k_{s}I_{i}(\bm{r_{i}} \cdot \bm{v})^s
\end{equation}

Wadą modelu Phonga jest ,,ucinanie się" odblasku lub jego zanik gdy kąt pomiędzy obserwatorem a wektorem odbicia jest większy niż 90 stopni, wtedy $\vecbm{r} \cdot \vecbm{v}$ jest ujemny, co doprowadza do zaniku komponentu odbicia lustrzanego (iloczyn skalarny ogranicza się od dołu 0). Problem rozwiązany jest w modelu Blinna-Phonga w którym oblicza się tzw. halfway vector (rysunek~\ref{fig:phongHVec}) $\bm{h} = \frac{\bm{l} + \bm{v}}{\|\bm{l + v}\|}$ i zamienia się $\bm{r} \cdot \bm{v}$ na $\bm{n} \cdot \bm{h}$~\cite{phongOpenGL}.

\begin{equation}
    I_{Blinn-Phong} = I_{a}k_{a} + k_{d}I_{i}(\bm{n} \cdot \bm{l}) + k_{s}I_{i}(\bm{n} \cdot \bm{h})^s
\end{equation}

\begin{figure}[tbph]
    \centering
    \includegraphics[width=1\textwidth]{phongVSblinphong} 
    \caption{Prównanie modeli Phonga i Blinna-Phonga~\cite{phongOpenGL}}
    \label{fig:phongBlinn}
\end{figure}

\begin{figure}[tbph]
    \centering
    \includegraphics[width=0.8\textwidth]{halfwayVector} 
    \caption{Rysunek przedstawia ,,halfway vector"~\cite{phongOpenGL}}
    \label{fig:phongHVec}
\end{figure}

\section{Model oświetlenia globalnego}

W modelach oświetlenia globalnego (ang. \textit{global illumination}) do obliczenia natężenia światła (koloru piksela) w danym punkcie bierze się pod uwagę źródło światła jak i otoczenie - światło odbite od innych powierzchni. Efekty takie jak: realistyczne cienie, odbicia i obiekty transparentne są implementowane za pomocą algorytmów globalnego oświetlenia.
    

Algorytmy oświetlenia globalnego oparte są na tzw. równaniu renderingu (ang. \textit{rendering equation})~\cite{realTimeRendering}:
\begin{equation}
    \bm{L_{o}}(\bm{p}, \bm{v}) = \bm{L_{e}}(\bm{p}, \bm{v}) + \int_{\Omega}f(\bm{l}, \bm{v})\bm{L_{o}}(r(\bm{p}, \bm{l}), \bm{-l})(\bm{n} \cdot \bm{l})^{+}d\bm{l}
\label{eq:rendering}
\end{equation}
%
gdzie:
\begin{itemize}
    \item $\bm{L_{o}(\bm{p}, \bm{v})}$ - światło wychodzące od punktu $p$ w kierunku $\bm{v}$ (kierunek kamery).
    \item $\bm{L_{e}(\bm{p}, \bm{v})}$ - światło emitowane przez powierzchnię w punkcie $p$ w kierunku $\bm{v}$.
    \item $\Omega$ - to powierzchnia półsfery, która znajduje się nad punktem dla którego oblicza się wychodzące światło. Całkowanie po tej półsferze odpowiada sumowaniu wkładu światła docierającego ze wszystkich możliwych kierunków.
    \item $f(\bm{l}, \bm{v})$ - to tzw. dwukierunkowa funkcja rozkładu odbicia (ang. \textit{bidirectional reflectance distribution function} dalej BRDF). BRDF to funkcja opisująca w jaki sposób światło nadchodzące z kierunku $\bm{l}$ jest odbijane w kierunku $\bm{v}$ przez dany materiał.
    \item $\bm{L_{o}}(r(\bm{p}, \bm{l}), \bm{-l})$ - przychodzące światło. $p$ to punkt na powierzchni, a $\bm{l}$ to kierunek. Funkcja $r(p, \bm{l})$ zwraca punkt przecięcia się promienia, którego $p$ to punkt początkowy, a $\bm{l}$ kierunek. Komponent ten oznacza, że światło przychodzące do punktu $p$ jest światłem wychodzącym od innego punktu.
    \item $(\bm{n} \cdot \bm{l})^{+}$ - iloczyn skalarny wektora normalnego powierzchni z wektorem kierunku światła, ponieważ oba wektory są znormalizowane wynikiem jest cosinus kąta pomiędzy tymi wektorami. $+$ oznacza branie tylko dodatnich wyników. 
\end{itemize}


Równanie \ref{eq:rendering} jest niemożliwe do rozwiązania analitycznego (poza bardzo prostymi scenami), ponieważ aby rozwiązać równanie dla punktu $p$ trzeba znać wynik dla punktu wcześniejszego $p'$, powstaje nieskończona rekurencja na nieskończonej liczbie kierunków z których światło dochodzi do punktu $p$.

Równanie renderingu rozwiązuje się metodami numerycznymi np. metodą Monte--Carlo.


Aby przybliżyć wynik całki w równaniu \ref{eq:rendering} używając metodę Monte--Carlo należy wylosować $N$ próbek i obliczyć średnią wartość funkcji pod całką.

\begin{algorithm}[H]
    \caption{Estymacja Monte Carlo koloru piksela (antyaliasing)}
    \label{alg:monte_carlo_pixel}
    \begin{algorithmic}[1] % [1] dodaje numerację linii
        \STATE $\bm{totalColor} \gets (0, 0, 0)$
        \FOR{$i = 0$ \TO $raysPerPixel$}
            \STATE $\bm{\epsilon} \gets \text{RandomVec3}(seed, -0.001, 0.001)$
            \STATE $\bm{d}_{jitter} \gets \text{normalize}(\bm{ray.direction} + \bm{\epsilon})$
            \STATE $jitteredRay \gets \text{Ray}(\bm{ray.origin}, \bm{d}_{jitter})$
            \STATE $\bm{totalColor} \gets \bm{totalColor} + \text{TraceRay}(jitteredRay, seed)$
        \ENDFOR
        \STATE $\bm{finalColor} \gets \bm{totalColor} / \text{float}(raysPerPixel)$
    \end{algorithmic}
\end{algorithm}


Przykładem BRDF, który jest wykorzystywany w implementacji oświetlenia globalnego (jak i lokalnego) jest model Cook-Torrance \cite{cookTorrance}. Model bazuje na tzw. \textit{microfacet theory}, polega ona na spostrzeżeniu, że w rzeczywistości nie ma idealnie gładkich powierzchni (najbliżej jest lustro), złożone są z "mikropowierzchni", które skierowane są pod różnym kątem. Poziom nierówności powierzchni opisuje się parametrem chropowatości (ang.\textit{roughness}). 

\begin{figure}[tbph]
    \centering
    \includegraphics[width=0.8\textwidth]{MIcrofacet} 
    \caption{Rysunek przedstawia mikropowierzchnie.~\cite{pbrbook}}
    \label{fig:microfacetPic}
\end{figure}

Funkcja BRDF modelu Cook-Torrance dla odbić lustrzanych\cite{ue4}: 

\begin{equation}
    \label{eq:cookTorranceEq}
    f_{cookTorrance} = \frac{D(\bm{n}, \bm{h}, \alpha)F(\theta)G(\bm{n}, \bm{v}, \bm{l})}{4(\bm{n} \cdot \bm{l})(\bm{n} \cdot \bm{v})}
\end{equation}

gdzie:
\begin{itemize}
    \item $D$ - Funkcja rozkładu mikropowierzchni. Określa ona jak dużo ścianek skierowana jest w taki sposób aby odbić nadchodzący promień w stronę kamery.
    \item $F$ - Współczynnik Fresnela, określa stosunek światła odbitego do światła załamanego.
    \item $G$ - Funkcja określająca atenuacje (tłumienie) światła wynikające z nakładania się mikropowierzchni na siebie (mikropowierzchnie mogą się zasłaniać).
    \item $\bm{n}$ - wektor normalny.
    \item $\bm{l}$ - wektor do źródła światła. W tym przypadku jest to kierunek z którego przychodzi światło, tj. wektor odbicia.
    \item $\bm{v}$ - wektor do kamery.
\end{itemize}


Wybór funkcji $D$ i $G$ może być różny, najpopularniejszą funkcją $D$ jest funkcja Trowbridge-Reitz (GGX - ,,Ground Glass Unknown"~\cite{GGX}), a $G$ to funkcja Smitha.

\begin{align}
    D_{GGX}(\bm{n}, \bm{h}, \alpha) &= \frac{\alpha^2}{\pi((\bm{n} \cdot \bm{h})^2(\alpha^2 - 1) + 1)^2} \\[10pt]
    G(\bm{n}, \bm{v}, \bm{l}) &= G_1(\bm{n}, \bm{v}) \cdot G_1(\bm{n}, \bm{l}) \\[10pt]
    G_1(\bm{n}, \bm{v}) &= \frac{\bm{n} \cdot \bm{v}}{(\bm{n} \cdot \bm{v})(1 - k) + k}
\end{align}

gdzie: 

\begin{itemize}
    \item $\alpha$ to parametr materiału ,,chropowatość" (ang. \textit{roughness}).
    \item $k$ = $\frac{(\alpha + 1)^2}{8}$
\end{itemize}

Współczynnik Fresnela w większości przypadków przybliża się tzw. przybliżeniem Schlick'a\cite{Schlick}.
\begin{equation}
    F_{Schlick} = F_0 + (1 - F_0)(1 - \cos(\theta))^5
\end{equation}

$F_0$ to współczynnik odbicia światła padającego prostopadle do powierzchni. W silniku przyjmuje się bazową wartość dla dielektryków (niemetale pomijając wyjątki) $f_0 = 0.04$. Jeśli obiekt jest metalem, to wartość $F_0$ obliczana jest według wzoru:

\begin{align}
    \bm{F}_{\text{dielectric}} &= 0,04 \\
    \bm{F}_0 &= \text{lerp}(\bm{F}_{\text{dielectric}}, \bm{\text{material}}.{\text{albedo}}, \bm{\text{material}}.{\text{metalness}})
\end{align}

Wzór \ref{eq:cookTorranceEq} służy do opisania odbić lustrzanych (ang. specular reflection). Do pełnego modelu brakuje jeszcze części opisującej światło rozproszone, np. modelu lamberta. 

\begin{equation}
    \bm{f_{\text{diffuse}}} = \frac{\sigma}{\pi}
\end{equation}
Gdzie:
$\sigma$ - to albedo (kolor) obiektu.
%
Razem:
\begin{equation}
    \bm{f}(\bm{v}, \bm{l}) = \bm{f}_{\text{diffuse}}(\bm{v}, \bm{l}) + \bm{f}_{\text{specular}}(\bm{v}, \bm{l})
\end{equation}

\begin{figure}[tbph]
    \centering
    \includegraphics[width=1\textwidth]{WholeModel} 
    \caption{Obrazek przedstawia cały model oświetlenia~\cite{cookTorranceModel}}
    \label{fig:lightModel}
\end{figure}

\begin{figure}[tbph]
    \centering
    \includegraphics[width=1\textwidth]{Shadows} 
    \caption{Przykład globalnego oświetlenia - miękkie cienie. Widać również wpływ sfer na kolor podłoża.}
    \label{fig:shadows}
\end{figure}
\chapter{Przedstawienie silnika}
\label{chap:engine}

\section{Biblioteki}
\section{Opis potoku graficznego}
\section{Implementacja algorytmów}
\chapter*{Podsumowanie}
\addcontentsline{toc}{chapter}{Podsumowanie}
\markboth{Podsumowanie}{Podsumowanie}
\label{chap:podsumowanie}

\section{Testy Wydajności}
\section{Zakończenie}

\printbibliography[heading=bibintoc]
\end{document}